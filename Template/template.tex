\documentclass[twoside]{amsart}
\usepackage{fullpage}

\usepackage[T1]{fontenc}
\usepackage[utf8]{inputenc} 
\usepackage{lmodern}

% Set this to true for English, false for Slovene
\newif\ifenglish
\englishfalse  % Change to \englishtrue for English

\ifenglish
  % English language support
  \usepackage[english]{babel}
\else
  % Slovenian language support
  \usepackage[slovene]{babel}  % Commennt if Slovenian support is unavailable
\fi

\usepackage{hyperref}
\usepackage{amsmath,amssymb,amsfonts,mathtools}
\usepackage{graphicx}
\graphicspath{{./images/}}

\usepackage{fancyhdr}
% Define page style with numbers only on odd pages
\fancypagestyle{main}{%
    \fancyhf{}  % Clear all headers and footers
    \fancyfoot[CO]{\thepage}  % Page number in center of odd pages only
    \renewcommand{\headrulewidth}{0pt}  % No header rule
    \renewcommand{\footrulewidth}{0pt}  % No footer rule
}
% Redefine plain style (used by \maketitle) to have no page number
\fancypagestyle{plain}{%
  \fancyhf{}
  \renewcommand{\headrulewidth}{0pt}
  \renewcommand{\footrulewidth}{0pt}
}

\linespread{1.2}

% Custom commands
\newcommand{\N}{\mathbb{N}}
\newcommand{\R}{\mathbb{R}}
\providecommand{\abs}[1]{\left\lvert #1 \right\rvert}
\newcommand{\probP}{\mathrm{I\kern-0.15em P}}
\newcommand{\E}{\mathbb{E}}

% Custom commands for proof pages
\newcommand{\proofpage}{\clearpage\ifodd\value{page}\null\clearpage\fi}
\newcommand{\statementpage}{\clearpage\ifodd\value{page}\else\null\clearpage\fi}

% Theorem environments - upright text
\theoremstyle{definition}
\ifenglish
  \newtheorem{definition}{Definition}[section]
  \newtheorem{example}[definition]{Example}
  \newtheorem{remark}[definition]{Remark}
  \renewcommand\endexample{\hfill$\diamondsuit$}
\else
  \newtheorem{definicija}{Definicija}[section]
  \newtheorem{primer}[definicija]{Primer}
  \newtheorem{opomba}[definicija]{Opomba}
  \renewcommand\endprimer{\hfill$\diamondsuit$}
\fi

% Theorem environments - italic text
\theoremstyle{plain}
\ifenglish
  \newtheorem{lemma}[definition]{Lemma}
  \newtheorem{theorem}[definition]{Theorem}
  \newtheorem{proposition}[definition]{Proposition}
  \newtheorem{corollary}[definition]{Corollary}
\else
  \newtheorem{lema}[definicija]{Lema}
  \newtheorem{izrek}[definicija]{Izrek}
  \newtheorem{trditev}[definicija]{Trditev}
  \newtheorem{posledica}[definicija]{Posledica}
\fi

\newcommand\Vtextvisiblespace[1][.3em]{%
\mbox{\kern.06em\vrule height.3ex}%
\vbox{\hrule width#1}%
\hbox{\vrule height.3ex}}

\title{Naslov dokumenta}
\author{Ime Priimek}
\date{\today}

\begin{document}
\pagestyle{main}

\maketitle
\thispagestyle{empty}

\tableofcontents
\newpage

\section{Uvod}

Predloga, kjer so definicije, izreki in posledice na sprednjih straneh 
(lihe številke), dokazi in primeri pa na zadnjih straneh (sode številke).

\statementpage

\section{Osnovni koncepti}

\begin{definicija}
  To je primer definicije, ki se bo pojavila na sprednji strani.
\end{definicija}

\begin{izrek}\label{izr:primer}
  To je primer izreka, ki se bo pojavil na sprednji strani.
\end{izrek}

\begin{posledica}
  To je primer posledice, ki se bo pojavila na sprednji strani.
\end{posledica}

\proofpage

\begin{proof}[Dokaz izreka \ref{izr:primer}]
  To je dokaz izreka, ki se bo pojavil na zadnji strani.
  \[
      \E[X] = \sum x \probP(x).
  \]
\end{proof}

\begin{primer}
  To je primer, ki se bo pojavil na zadnji strani skupaj z dokazi.
  \begin{align*}
      B(\mathbf{0}, 1) &= \{(x,y) \in \R^2 \mid d(\mathbf{0}, (x,y)) < 1\} \\
      &= \{(x,y) \in \R^2 \mid x^2 + y^2 < 1\}.
  \end{align*}
  Prikazana na sliki~\ref{img:alpha1},
\end{primer}


\statementpage

\section{Napredne teme}

\begin{definicija}[Še ena definicija]
  Definicije, izreki in posledice se vedno pojavljajo na lihih straneh (spredaj).
\end{definicija}

\begin{lema}\label{lem:pomozna}
  To je pomožna lema.
\end{lema}

\begin{trditev}
  To je trditev, ki bo na sprednji strani.
\end{trditev}

\proofpage

\begin{proof}[Dokaz leme~\ref{lem:pomozna}]
  Dokaz pomožne leme se pojavi na zadnji strani.
\end{proof}

\begin{opomba}
  Opombe lahko postavite na katerokoli stran, odvisno od konteksta.
\end{opomba}

\begin{primer}
  Primeri se pojavljajo na zadnjih straneh skupaj z dokazi.

\begin{figure}[ht]
    \centering
    \includegraphics[width=0.3\textwidth]{alpha1.jpg}
    \caption{Odprta krogla v ravnini.} 
    \label{img:alpha1} % After caption
\end{figure}

\end{primer}

\end{document}